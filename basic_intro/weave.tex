\documentclass[12pt,a4paper]{article}

\usepackage[a4paper,text={16.5cm,25.2cm},centering]{geometry}
\usepackage{lmodern}
\usepackage{amssymb,amsmath}
\usepackage{bm}
\usepackage{graphicx}
\usepackage{microtype}
\usepackage{hyperref}
\setlength{\parindent}{0pt}
\setlength{\parskip}{1.2ex}

\hypersetup
       {   pdfauthor = { Yingbo Ma, Chris Rackauckas },
           pdftitle={ Classical Physics Models },
           colorlinks=TRUE,
           linkcolor=black,
           citecolor=blue,
           urlcolor=blue
       }

\title{ Classical Physics Models }

\author{ Yingbo Ma, Chris Rackauckas }


\usepackage{upquote}
\usepackage{listings}
\usepackage{xcolor}
\lstset{
    basicstyle=\ttfamily\footnotesize,
    upquote=true,
    breaklines=true,
    breakindent=0pt,
    keepspaces=true,
    showspaces=false,
    columns=fullflexible,
    showtabs=false,
    showstringspaces=false,
    escapeinside={(*@}{@*)},
    extendedchars=true,
}
\newcommand{\HLJLt}[1]{#1}
\newcommand{\HLJLw}[1]{#1}
\newcommand{\HLJLe}[1]{#1}
\newcommand{\HLJLeB}[1]{#1}
\newcommand{\HLJLo}[1]{#1}
\newcommand{\HLJLk}[1]{\textcolor[RGB]{148,91,176}{\textbf{#1}}}
\newcommand{\HLJLkc}[1]{\textcolor[RGB]{59,151,46}{\textit{#1}}}
\newcommand{\HLJLkd}[1]{\textcolor[RGB]{214,102,97}{\textit{#1}}}
\newcommand{\HLJLkn}[1]{\textcolor[RGB]{148,91,176}{\textbf{#1}}}
\newcommand{\HLJLkp}[1]{\textcolor[RGB]{148,91,176}{\textbf{#1}}}
\newcommand{\HLJLkr}[1]{\textcolor[RGB]{148,91,176}{\textbf{#1}}}
\newcommand{\HLJLkt}[1]{\textcolor[RGB]{148,91,176}{\textbf{#1}}}
\newcommand{\HLJLn}[1]{#1}
\newcommand{\HLJLna}[1]{#1}
\newcommand{\HLJLnb}[1]{#1}
\newcommand{\HLJLnbp}[1]{#1}
\newcommand{\HLJLnc}[1]{#1}
\newcommand{\HLJLncB}[1]{#1}
\newcommand{\HLJLnd}[1]{\textcolor[RGB]{214,102,97}{#1}}
\newcommand{\HLJLne}[1]{#1}
\newcommand{\HLJLneB}[1]{#1}
\newcommand{\HLJLnf}[1]{\textcolor[RGB]{66,102,213}{#1}}
\newcommand{\HLJLnfm}[1]{\textcolor[RGB]{66,102,213}{#1}}
\newcommand{\HLJLnp}[1]{#1}
\newcommand{\HLJLnl}[1]{#1}
\newcommand{\HLJLnn}[1]{#1}
\newcommand{\HLJLno}[1]{#1}
\newcommand{\HLJLnt}[1]{#1}
\newcommand{\HLJLnv}[1]{#1}
\newcommand{\HLJLnvc}[1]{#1}
\newcommand{\HLJLnvg}[1]{#1}
\newcommand{\HLJLnvi}[1]{#1}
\newcommand{\HLJLnvm}[1]{#1}
\newcommand{\HLJLl}[1]{#1}
\newcommand{\HLJLld}[1]{\textcolor[RGB]{148,91,176}{\textit{#1}}}
\newcommand{\HLJLs}[1]{\textcolor[RGB]{201,61,57}{#1}}
\newcommand{\HLJLsa}[1]{\textcolor[RGB]{201,61,57}{#1}}
\newcommand{\HLJLsb}[1]{\textcolor[RGB]{201,61,57}{#1}}
\newcommand{\HLJLsc}[1]{\textcolor[RGB]{201,61,57}{#1}}
\newcommand{\HLJLsd}[1]{\textcolor[RGB]{201,61,57}{#1}}
\newcommand{\HLJLsdB}[1]{\textcolor[RGB]{201,61,57}{#1}}
\newcommand{\HLJLsdC}[1]{\textcolor[RGB]{201,61,57}{#1}}
\newcommand{\HLJLse}[1]{\textcolor[RGB]{59,151,46}{#1}}
\newcommand{\HLJLsh}[1]{\textcolor[RGB]{201,61,57}{#1}}
\newcommand{\HLJLsi}[1]{#1}
\newcommand{\HLJLso}[1]{\textcolor[RGB]{201,61,57}{#1}}
\newcommand{\HLJLsr}[1]{\textcolor[RGB]{201,61,57}{#1}}
\newcommand{\HLJLss}[1]{\textcolor[RGB]{201,61,57}{#1}}
\newcommand{\HLJLssB}[1]{\textcolor[RGB]{201,61,57}{#1}}
\newcommand{\HLJLnB}[1]{\textcolor[RGB]{59,151,46}{#1}}
\newcommand{\HLJLnbB}[1]{\textcolor[RGB]{59,151,46}{#1}}
\newcommand{\HLJLnfB}[1]{\textcolor[RGB]{59,151,46}{#1}}
\newcommand{\HLJLnh}[1]{\textcolor[RGB]{59,151,46}{#1}}
\newcommand{\HLJLni}[1]{\textcolor[RGB]{59,151,46}{#1}}
\newcommand{\HLJLnil}[1]{\textcolor[RGB]{59,151,46}{#1}}
\newcommand{\HLJLnoB}[1]{\textcolor[RGB]{59,151,46}{#1}}
\newcommand{\HLJLoB}[1]{\textcolor[RGB]{102,102,102}{\textbf{#1}}}
\newcommand{\HLJLow}[1]{\textcolor[RGB]{102,102,102}{\textbf{#1}}}
\newcommand{\HLJLp}[1]{#1}
\newcommand{\HLJLc}[1]{\textcolor[RGB]{153,153,119}{\textit{#1}}}
\newcommand{\HLJLch}[1]{\textcolor[RGB]{153,153,119}{\textit{#1}}}
\newcommand{\HLJLcm}[1]{\textcolor[RGB]{153,153,119}{\textit{#1}}}
\newcommand{\HLJLcp}[1]{\textcolor[RGB]{153,153,119}{\textit{#1}}}
\newcommand{\HLJLcpB}[1]{\textcolor[RGB]{153,153,119}{\textit{#1}}}
\newcommand{\HLJLcs}[1]{\textcolor[RGB]{153,153,119}{\textit{#1}}}
\newcommand{\HLJLcsB}[1]{\textcolor[RGB]{153,153,119}{\textit{#1}}}
\newcommand{\HLJLg}[1]{#1}
\newcommand{\HLJLgd}[1]{#1}
\newcommand{\HLJLge}[1]{#1}
\newcommand{\HLJLgeB}[1]{#1}
\newcommand{\HLJLgh}[1]{#1}
\newcommand{\HLJLgi}[1]{#1}
\newcommand{\HLJLgo}[1]{#1}
\newcommand{\HLJLgp}[1]{#1}
\newcommand{\HLJLgs}[1]{#1}
\newcommand{\HLJLgsB}[1]{#1}
\newcommand{\HLJLgt}[1]{#1}


\begin{document}

\maketitle

If you're getting some cold feet to jump in to DiffEq land, here are some handcrafted differential equations mini problems to hold your hand along the beginning of your journey.

\subsection{First order linear ODE}
\paragraph{Radioactive Decay of Carbon-14}
\[
f(t,u) = \frac{du}{dt}
\]
The Radioactive decay problem is the first order linear ODE problem of an exponential with a negative coefficient, which represents the half-life of the process in question. Should the coefficient be positive, this would represent a population growth equation.


\begin{lstlisting}
(*@\HLJLk{using}@*) (*@\HLJLn{OrdinaryDiffEq}@*)(*@\HLJLp{,}@*) (*@\HLJLn{Plots}@*)
(*@\HLJLnf{gr}@*)(*@\HLJLp{()}@*)

(*@\HLJLcs{{\#}Half-life}@*) (*@\HLJLcs{of}@*) (*@\HLJLcs{Carbon-14}@*) (*@\HLJLcs{is}@*) (*@\HLJLcs{5,730}@*) (*@\HLJLcs{years.}@*)
(*@\HLJLn{C\ensuremath{\_1}}@*) (*@\HLJLoB{=}@*) (*@\HLJLnfB{5.730}@*)

(*@\HLJLcs{{\#}Setup}@*)
(*@\HLJLn{u\ensuremath{\_0}}@*) (*@\HLJLoB{=}@*) (*@\HLJLnfB{1.0}@*)
(*@\HLJLn{tspan}@*) (*@\HLJLoB{=}@*) (*@\HLJLp{(}@*)(*@\HLJLnfB{0.0}@*)(*@\HLJLp{,}@*) (*@\HLJLnfB{1.0}@*)(*@\HLJLp{)}@*)

(*@\HLJLcs{{\#}Define}@*) (*@\HLJLcs{the}@*) (*@\HLJLcs{problem}@*)
(*@\HLJLnf{radioactivedecay}@*)(*@\HLJLp{(}@*)(*@\HLJLn{u}@*)(*@\HLJLp{,}@*)(*@\HLJLn{p}@*)(*@\HLJLp{,}@*)(*@\HLJLn{t}@*)(*@\HLJLp{)}@*) (*@\HLJLoB{=}@*) (*@\HLJLoB{-}@*)(*@\HLJLn{C\ensuremath{\_1}}@*)(*@\HLJLoB{*}@*)(*@\HLJLn{u}@*)

(*@\HLJLcs{{\#}Pass}@*) (*@\HLJLcs{to}@*) (*@\HLJLcs{solver}@*)
(*@\HLJLn{prob}@*) (*@\HLJLoB{=}@*) (*@\HLJLnf{ODEProblem}@*)(*@\HLJLp{(}@*)(*@\HLJLn{radioactivedecay}@*)(*@\HLJLp{,}@*)(*@\HLJLn{u\ensuremath{\_0}}@*)(*@\HLJLp{,}@*)(*@\HLJLn{tspan}@*)(*@\HLJLp{)}@*)
(*@\HLJLn{sol}@*) (*@\HLJLoB{=}@*) (*@\HLJLnf{solve}@*)(*@\HLJLp{(}@*)(*@\HLJLn{prob}@*)(*@\HLJLp{,}@*)(*@\HLJLnf{Tsit5}@*)(*@\HLJLp{())}@*)

(*@\HLJLcs{{\#}Plot}@*)
(*@\HLJLnf{plot}@*)(*@\HLJLp{(}@*)(*@\HLJLn{sol}@*)(*@\HLJLp{,}@*)(*@\HLJLn{linewidth}@*)(*@\HLJLoB{=}@*)(*@\HLJLni{2}@*)(*@\HLJLp{,}@*)(*@\HLJLn{title}@*) (*@\HLJLoB{=}@*)(*@\HLJLs{"{}Carbon-14}@*) (*@\HLJLs{half-life"{}}@*)(*@\HLJLp{,}@*) (*@\HLJLn{xaxis}@*) (*@\HLJLoB{=}@*) (*@\HLJLs{"{}Time}@*) (*@\HLJLs{in}@*) (*@\HLJLs{thousands}@*) (*@\HLJLs{of}@*) (*@\HLJLs{years"{}}@*)(*@\HLJLp{,}@*) (*@\HLJLn{yaxis}@*) (*@\HLJLoB{=}@*) (*@\HLJLs{"{}Percentage}@*) (*@\HLJLs{left"{}}@*)(*@\HLJLp{,}@*) (*@\HLJLn{label}@*) (*@\HLJLoB{=}@*) (*@\HLJLs{"{}Numerical}@*) (*@\HLJLs{Solution"{}}@*)(*@\HLJLp{)}@*)
(*@\HLJLnf{plot!}@*)(*@\HLJLp{(}@*)(*@\HLJLn{sol}@*)(*@\HLJLoB{.}@*)(*@\HLJLn{t}@*)(*@\HLJLp{,}@*) (*@\HLJLn{t}@*)(*@\HLJLoB{->}@*)(*@\HLJLnf{exp}@*)(*@\HLJLp{(}@*)(*@\HLJLoB{-}@*)(*@\HLJLn{C\ensuremath{\_1}}@*)(*@\HLJLoB{*}@*)(*@\HLJLn{t}@*)(*@\HLJLp{),}@*)(*@\HLJLn{lw}@*)(*@\HLJLoB{=}@*)(*@\HLJLni{3}@*)(*@\HLJLp{,}@*)(*@\HLJLn{ls}@*)(*@\HLJLoB{=:}@*)(*@\HLJLn{dash}@*)(*@\HLJLp{,}@*)(*@\HLJLn{label}@*)(*@\HLJLoB{=}@*)(*@\HLJLs{"{}Analytical}@*) (*@\HLJLs{Solution"{}}@*)(*@\HLJLp{)}@*)
\end{lstlisting}

\begin{lstlisting}
Error: ArgumentError: Package OrdinaryDiffEq not found in current path:
- Run (*@{{\textasciigrave}}@*)import Pkg; Pkg.add((*@{"{}}@*)OrdinaryDiffEq(*@{"{}}@*))(*@{{\textasciigrave}}@*) to install the OrdinaryDiffEq
 package.
\end{lstlisting}


\subsection{Second Order Linear ODE}
\paragraph{Simple Harmonic Oscillator}
Another classical example is the harmonic oscillator, given by

\[
\ddot{x} + \omega^2 x = 0
\]
with the known analytical solution


\begin{align*}
x(t) &= A\cos(\omega t - \phi) \\
v(t) &= -A\omega\sin(\omega t - \phi),
\end{align*}
where

\[
A = \sqrt{c_1 + c_2} \qquad\text{and}\qquad \tan \phi = \frac{c_2}{c_1}
\]
with $c_1, c_2$ constants determined by the initial conditions such that $c_1$ is the initial position and $\omega c_2$ is the initial velocity.

Instead of transforming this to a system of ODEs to solve with \texttt{ODEProblem}, we can use \texttt{SecondOrderODEProblem} as follows.


\section{Simple Harmonic Oscillator Problem}
using OrdinaryDiffEq, Plots

\#Parameters \ensuremath{\omega} = 1

\#Initial Conditions x\ensuremath{\_0} = [0.0] dx\ensuremath{\_0} = [\ensuremath{\pi}/2]

tspan = (0.0, 2\ensuremath{\pi})



\end{document}
